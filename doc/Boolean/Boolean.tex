\documentclass{article}

\begin{document}

\title{Boolean Logic}

\author{Peter Mills}

\maketitle

\section{Introduction}

Let $P$ be a binary proposition. 
$P$ may take on two values: it is either true or false, $0$ or $1$;
$P \in \lbrace 0, ~ 1 \rbrace$.

Here we define   operators for two propositions, $X$ and $Y$:
\begin{tabular}{ll}
	Name & Symbol & Example \\
	\hline
	AND & $\land$ & $X \land Y$ \\
	OR & $\lor$ & $X \lor Y$\\
	NOT & $\lnot$ $\lnot X$ \\
	NAND & - & $\lnot (X \land Y)$\\
	NOR & - & $\lnot (X \lor Y)$ \\
	if then & $\rightarrow$ & $X \rightarrow Y$ \\
	if-and-only-if & $\iff$ & $X \iff Y$ \\
	exclusive OR (XOR) & \oplus & $X \oplus Y$ \\ 
	\hline
\end{tabular}
All operators return another binary proposition.

Here is the {\it truth table} for the AND operator:
\begin{tabular}{lll}
	$X$ & $Y$ & $X \land Y$ \\
	0 & 0 & 0 \\
	0 & 1 & 0 \\
	1 & 0 & 0 \\
	1 & 1 & 1 \\
	\hline
\end{tabular}

Note that the logical OR is different from the English ``or''.
In English speech, ``or'' is the equivalent to the exclusive OR (
XOR), defined below.
\begin{tabular}{lll}
	$X$ & $Y$ & $X \lor Y$ \\
	0 & 0 & 0 \\
	0 & 1 & 1 \\
	1 & 0 & 1 \\
	1 & 1 & 1 \\
	\hline
\end{tabular}

NOT is logical negation:
\begin{tabular}{lll}
	$X$ & $\lnot X$ \\
	 1 & 0 \\
	 0 & 1 \\
	\hline
\end{tabular}

Truth tables for NAND and NOR may be derived from their definitions.
All operators may be defined in terms of $\lnot$ and either
NAND or NOR.

If $X$ then $Y$ ($X \rightarrow Y$) is defined:
\begin{tabular}{lll}
	$X$ & $Y$ & $X \rightarrow Y$ \\
	0 & 0 & 1 \\
	0 & 1 & 1 \\
	1 & 0 & 0 \\
	1 & 1 & 1 \\
	\hline
\end{tabular}
This is equivalent to $\lnot X \lor Y$.
$X$ if-and-only-if $Y$ is equivalent to a negated XOR whose truth table is
as follows:
\begin{tabular}{lll}
	$X$ & $Y$ & $X \oplus Y$ \\
	0 & 0 & 0 \\
	0 & 1 & 1 \\
	1 & 0 & 1 \\
	1 & 1 & 0 \\
	\hline
\end{tabular}

We can write:
\begin{equation}
	(A \liff B) \rightarrow \lnot (A \oplus B)
	\label{iffimpliesnotxor}
\end{equation}
This is called a {\it tautology} which is a {\it logical expression} that
always evaluates to 1.
More generally, it is also a {\it theorem} within the system of Boolean
logic.

\subsection{Exercises}

\begin{enumerate}
	\item Write all the operators in terms of NOT and NOR.
	\item Write all the operators in terms of NOT and NAND.
	\item Derive the truth table for NOR, NAND and $\iff$.
	\item Show that \ref{iffimpliesnotxor} is a tautology.
\end{enumerate}

\section{List of tautologies}

Prove the following theorems:
\begin{equation}
$\lnot(A \and B}) \iff (\lnot A \lor \lnot B)
\end{equation}
Distributive property:
\begin{eqnarray}
	(A \and (B \lor C)) \iff ((A \land B) \lor (A \land C) \\
	(A \or (B \land C)) \iff ((A \lor B) \land (A \lor C) \\
\end{eqnarray}

\begin{equation}
	((A \rightarrow B) \land (B \rightarrow A) \iff (A \iff B))
\end{equation}

\section{Gates}

Any boolean expression can be represented as a {\it logic circuit} 
using {\it logic gates}.


\section{Notation}

\end{document}

