\documentclass{article}

\title{A primer on data structures for machine learning}

\begin{document}

\section{Introduction}

So you've decided to move beyond canned algorithms and code you own machine
learning methods.
Maybe you've got an idea for a cool new way of clustering data or maybe you
are frustrated by the limitations in you favourite statistical classification
package.
In either case, the better your knowledge of data structures and algorithms,
the easier you'll have it when it comes time to code up.

I don't think the data structures used in machine learning are significantly
different than those used in other areas of software development.
However, because of the size and difficulty of many of the problems, having
a really solid handle on the basics is essential.
Also, because machine learning is a very mathematical field, one should have in
mind how data structures can be used to solve mathematical problems and 
as mathematical objects in their own right.

There are two ways to classify data structures: by their implementation and by
their operation.
By implementation, I mean the nuts and bolts of how they are programmed and
the actual storage patterns.
How they look on the outside is less important than what's going on under the
hood.
For data structures classed by operation, it is the opposite: their external
appearance and operation is more important than how they are implemented
and in fact they can usually be implemented using a number of different internal
representations.

\section{Arrays}

I'm not kidding. The basic array is the most important data structure in
machine learning and there is more to this bread-and-butter structure than
you might think.
Arrays are so important because they are used in linear algebra: the most 
useful and powerful mathematical tool at your disposal.
So the most common types will be the one- and two-dimensional variety,
corresponding to vectors and matrices respectively, but you will occasionally
encounter three- or four-dimensional either for higher ranked tensors or to
group examples of the former.

\section{Extensible arrays}

In most cases arrays can be allocated to a fixed size at run time or you can
calculate a reliable upper bound.
In those cases where you need your arrays to expand indefinitely, you can use 
an extensible array such as the vector class in the C++ standard template 
library (STL).
Regular arrays in Matlab are similar extensible and extensible arrays are the
basis of the entire Python language.

In an extensible array, there are two pieces of ``meta-data'' stored alongside
the actual data values. These are: the amount of storage space allocated to
the data structure and the actual size of the array.
As soon as the size of the array exceeds the storage space, a new space is
allocated that's twice the size, the values copied into it and the old array
is deleted.
This is an $O(n)$ operation, where $n$ is the size of the array, but since it
only happens occasionally, time to add a new value onto the end actually
amortizes to constant time, $O(1)$. It is a very flexible data structure with
fast average insertions and fast access, which is also constant in time.

\section{Linked list}

A linked list consists of several separately allocated {\it nodes}.
Each node contains a data value plus a pointer to the next node in the list.
Insertions, at constant time, are very efficient but accessing a value is
slow and often requires scanning through much of the list.
Linked lists are easy to splice together and split apart.
There are many variations: for instance insertions can be done at either 
the head or the tail; the list can be doubly-linked and there are many
similar data structures based on the same principle such as the 
binary tree, below.

Mainly I find linked lists useful for parsing lists of indeterminate length.
Afterwards they can be converted to fixed-length arrays for fast access.
For this reason I use a linked-list class that includes a method for conversion
to an array.

\section{Binary tree}

A binary tree is similar to a linked list except that each node has two 
pointers to subsequent nodes instead of just one.
The value in the {\it left child} is always less than the value in the 
{\it parent node} which in turn is smaller than that of the {\it right child}.
Thus, data in binary trees is automatically sorted.
Both insertion and access are efficient at $O(\log n)$ on average.
Like linked-lists they are easy to transform into arrays and this is the basis
for a tree-sort.

\subsection{Balanced trees}

If the data is already already sorted, binary trees are less
efficient at $O(n)$ worst case since the data will be laid out linearly as if 
it were a linked-list.
While the ordering in a binary tree is constrained, it is by no means unique
and the same list can be arranged in many different configurations depending
on the order in which it is inserted.
Thus there are several transformations that can be applied to a tree in order
to balance it.
{\it Self-balancing trees} perform these operations automatically in order to
keep access and insertion at an optimal average.

\section{Heap}

A heap is another hierarchical, ordered data structure similar to a tree except
instead of a horizonal ordering, it has a vertical ordering.
This ordering applies along the hierarchy, but not across it: the parent is
always larger than both its children but a node of higher rank is not 
necessarily larger any lower one that's not directly beneath it.

Both insertion and retrieval are performed by promotion. An element is
first inserted in the highest available position. Then it is compared with
its parent and promoted until it reaches the right rank.
To take an element off the heap, the larger of the two children is promoted
to the missing position, then the larger of those two children promoted and
so on until everything has trickled up the ranks.
Typically the highest ranking value at the top is pulled off the heap in order
to sort a list.
Unlike a tree, most heaps are simply stored in an array with the relationships
between elements only implicit.

\section{Stack}

A stack is defined as first in, last out. An element is {\it pushed} onto the 
top of the stack where it covers the previous element. The top element must be
{\it popped} off before any of the others can be accessed.
Stacks are mainly useful for parsing grammars and implementing computer 
languages. For instance, suppose you have a control language
that uses a special character to repeat
a previous option, except the language is recursive and the option is taken 
from the same or higher hierarchical level. 
This would best be implemented by a stack.

\section{Queu}

A queu is defined as first in, first out. Think of the line at the bank 
teller (for those of us still old enough to remember a time before internet
banking). Queus are useful in real time programming so that the program can
maintain a list of jobs to be processed. Consider an application to record
split times of athletes. You type in the bib number and hit enter--except wait:
in the time it took you to do that the next athlete behind has also passed.
So you type in a list of bib numbers of the nearest approaching athletes, 
then hit a separate key to register the next in the queu as having passed.

\section{Set}

A set consists of an un-ordered list of non-repeating elements.
Thus if you add an element that's already in there, there will be no change.
Since much of the mathematics of machine learning deals with sets they are
very useful data structures.

\section{Associative arrays}

In an associative array, there are two types of data which are stored in pairs:
the {\it key} and its associated {\it value}. 
The data structure is relational in nature: the value is addressed by its key.
Since much of the training data is relational
in nature, this type of data structure would seem ideally suited
to machine learning problems.
In practice, it's not used so much, in part because most associative arrays
are only one-dimensional whereas machine learning data is typically
multi-dimensional.

\section{Custom data structures}

The more problems you work on, the more likely you will encounter something
for which standard 

As you work on more problems, you are sure to encounter those for which the
standard recipe box does not contain optimal structures.
You will need to design your own data structure.

Consider a multi-class classifier which generalizes a binary classifier to
work with classification problems having more than two classes.
An obvious solution is bisection: recursively split the classes into
two groups.
You could use something similar to a binary tree to organize the binary
classifiers.
Except that a hierarchical solution is not the only method of solving for
multi-class.
Consider several partitions that are then used to solve for all the 
class probabilities simultaneously.
The most general solution would combine the two: thus each hierarchical
partition need not be binary but could be solved by a non-hierarchical 
multi-class classifier.

More complex data structures can also be composed of the basic structures.
Consider a sparse matrix class. In a sparse matrix, most of the elements
are zero and only the non-zero elements are stored.
We could store the position and value for each element as a triplet and
have a list of them in an extensible array.
The array is sorted by position to make locating each element faster.
This is the 3 by 3 identity:

\begin{equation}
I = \left [ \begin{array}{l}
	\lbrace 1, 1, 1. \rbrace\\
	\lbrace 2, 2, 1. \rbrace\\
	\lbrace 3, 3, 1. \rbrace
\end{array}
\right ]
\end{equation}

\section{Problems}

\begin{enumerate}
  \item How would you improve the example program?
  \item Do you think the example program should be split into two? If so, how would you split it?
  \item Encapsulate the vector multiplication routine in a subroutine called \verb/matrix_times_vector/. Design the calling syntax for the subroutine.
  \item Using \verb/struct/, \verb/typedef/ or \verb/class/, encapsulate both vectors and matrices into a pair of abstract types called \verb/vect/ and \verb/matrix/, respectively. Design an API for the types.
  \item Find at least three libraries online that do the above.
  \item Download and install the LIBSVM library. Consider the method 
	  \verb/Kernel:k_function/ on line 316 of ``svm.cpp''. What are the
	  advantages and disadvantages of the data structure used to hold 
	  vectors?
  \item How would you re-factor calculation of kernel functions in the LIBSVM library?
  \item What data structures would you use to implement the abstract data types, stack, qeue, set and associative array?
  \item Using a binary tree, design an associative array.
  \item Consider the vector type in LIBSVM. How can this be used to represent a sparse matrix? Contrast this with the sparse matrix class described above (the complete type can be found: github.com/peteysoft/libmsci/sparse/sparse.h). What are the advantages and disadvantages of each representation?
\end{enumerate}

\end{document}

