\documentclass{article}

\begin{document}

\title{Essentials of Linear Algebra}

\author{Peter Mills}

\maketitle

\section{Introduction}

Of all the tools in my box as a scientist and computer programmer, linear
algebra has been the most useful.
If a problem can be reduced to one in linear algebra, it is generally 
considered solved and efficient methods exist to solve even large linear
systems.
I have used linear algebra to solve systems of both ordinary and partial
differential equations.
It is used extensively in statistics and forms the basis of least squares
fitting.

....

Simply put, if you cannot do linear algebra, you cannot do science and you
cannot do machine learning.
This tutorial will introduce you to all the essentials you need do work with
systems of linear equations.

\section{Notation}

Consider the following system of linear equations:
\begin{eqnarray}
4 & = & x_1 + 2 x_2 + 3 x_3 \\
5 & = & 3 x_1 + 7 x_2 + 2 x_3 \\
0.5 & = & 2.1 x_1 + 5.3 x_2 + 4.7 x_3
\end{eqnarray}
Note that there are three equations 
and three unknowns: $x_1$, $x_2$ and $x_3$.
As a general rule, for a system of equations to be solvable, there should be
the same number of equations as unknowns.
This is not always the case: sometimes some symmetry in the equations allows
one to solve for extra unknowns and sometimes there is a redundancy in the
equations that make it impossible to solve for all the unkowns.

Lets rewrite this equation by assigning each of the coefficients 
and each of the constant values on the left-hand-side (LHS)  
to a variable:
\begin{eqnarray}
y_1 & = & a_{11} x_1 + a_{12} x_2 + a_{13} x_3 \\
y_2 & = & a_{21} x_1 + a_{22} x_2 + a_{23} x_3 \\
y_3 & = & a_{31} x_1 + a_{32} x_2 + a_{33} x_3
\end{eqnarray}
Note that each of the coefficients has the same variable name, $a$, but 
different subscripts.
The first subscript increments with the equation while the second increments
based on which $x$ the $a$ multiplies.
A similar system is used for the constants.

Lets rewrite this again, but for an arbitrarily large system:
\begin{eqnarray}
y_1 & = & a_{11} x_1 + a_{12} x_2 + a_{13} x_3 + ... + a_{1n} x_n \\
y_2 & = & a_{21} x_1 + a_{22} x_2 + a_{23} x_3 + ... + a_{2n} x_n \\
y_3 & = & a_{31} x_1 + a_{32} x_2 + a_{33} x_3 + ... + a_{3n} x_n \\ \nonumber
... \\
y_m & = & a_{m1} x_1 + a_{m2} x_2 + a_{m3} x_3 + ... + a_{mn} x_n
\end{eqnarray}
Note that there are $m$ equations and $n$ unknowns.
Now we rewrite it again using a more powerful technique called 
{\it summation notation}:
\begin{equation}
y_i = \sum_{j=1}^n a_{ij} x_j; ~ i=[1..m]
\end{equation}
Notice how much more compact this notation is yet nonetheless captures
everything contained in the previous two systems.
We can encapsulate this even more by {\it abstracting} the entire operation
as a type of multiplication:
\begin{equation}
\vec y = A \vec x
\label{matrix_vector_multiplication}
\end{equation}
where $A$ is an $m \times n$ {\it array}, $\vec y$ is a {\it vector} of length
$m$ and $\vec x$ is a vector of length $n$.
To bring this full circle, we write:
\begin{equation}
A = \left [ \begin{array}{lll}
1 & 2 & 3 \\
3 & 7 & 2 \\
2.1 & 5.3 & 4.7
\end{array}
\right ]
\end{equation}
and:
\begin{equation}
\vec y = \left [ \begin{array}{l}
4 \\
5 \\
0.5
\end{array}
\right ]
\end{equation}

Half the battle in mastering linear algebra comes in mastering the notation.
Usually when I am stuck in a problem I will move down to each of the less
abstract notations in turn, which, while more verbose, are also more clear.

\subsection{Example: proving the distributive property}

The same notation is used in Equation (\ref{matrix_vector_multiplication})
as for multiplication.
Indeed, it is even called multiplication: we are multiplying a matrix with
a vector.
There is a reason for this.
Matrix multiplication shares many of the same properties as scalar multiplication.
One of these is the distributive property:
\begin{equation}
A \vec x + A \vec y = A (\vec x + \vec y)
\end{equation}
We can use summation notation to prove this property:
\begin{equation}
\sum_j a_{ij} x_j + \sum_j a_{ij} y_j = \sum_j a_{ij} (x_j + y_j)
\end{equation}

\subsection{Matrix transpose}

One way in which matrix multiplication differs from scalar multiplication
is that it is not commutative.
In fact we can write:
\begin{equation}
A^T \vec x = \vec x A
\end{equation}
where $A^T$ is the {\it transpose} of matrix $A$.
Writing this out in summation notation makes it clear.
We simply swap the roles of $i$ and $j$:
\begin{equation}
\sum_j a_{ji} x_j
\end{equation}
Or alternatively:
\begin{equation}
\sum_i a_{ij} x_i
\end{equation}

\section{Solving systems of linear equations}

In order to solve for the system in (\ref{example_system}) we need to learn
one more notation system.
This one removes all the variables leaving only the constants and coefficients:
\begin{equation}
	\left [ \begin{array}{lll|l}
			4 & 1 & 2 & 3 & 4 \\
			5 & 3 & 7 & 2 & 5 \\
			0.5 & 2.1 & 5.3 & 4.7 & 0.5
	\end{array} \right ]
\end{equation}
There are three elementary operations we can do on the rows of this system.
As we would expect,
any operation done on a row of coefficients is repeated on the constant.
\begin{itemize}
	\item Multiply a row by a constant value
	\item Exchange rows
	\item Add two rows
\end{itemize}

\end{document}

