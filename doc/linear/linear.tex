\documentclass{article}

\begin{document}

\title{Essentials of Linear Algebra}

\author{Peter Mills}

\maketitle

\section{Introduction}

Of all the tools in my box as a scientist and computer programmer, linear
algebra has been the most useful.
If a problem can be reduced to one in linear algebra, it is generally 
considered solved and efficient methods exist to solve even large linear
systems.
I have used linear algebra to solve systems of both ordinary and partial
differential equations.
It is used extensively in statistics and forms the basis of least squares
fitting.

....

Simply put, if you cannot do linear algebra, you cannot do science and you
cannot do machine learning.
This tutorial will introduce you to all the essentials you need do work with
systems of linear equations.

\section{Notation}

\label{notation}

Consider the following system of linear equations:
\begin{eqnarray}
	x_1 + 2 x_2 + 3 x_3 & = & 4 \\ \nonumber
	3 x_1 + 7 x_2 + 2 x_3 & = & 5 \\ \nonumber
	2.1 x_1 + 5.3 x_2 + 4.7 x_3 & = & 0.5
	\label{example_system}
\end{eqnarray}
Note that there are three equations 
and three unknowns: $x_1$, $x_2$ and $x_3$.
As a general rule, for a system of equations to be solvable, there should be
the same number of equations as unknowns.
This is not always the case: sometimes some symmetry in the equations allows
one to solve for extra unknowns and sometimes there is a redundancy in the
equations that make it impossible to solve for all the unkowns.

Lets rewrite this equation by assigning each of the coefficients 
and each of the constant values on the left-hand-side (LHS)  
to a variable:
\begin{eqnarray}
	a_{11} x_1 + a_{12} x_2 + a_{13} x_3 & = & y_1 \\
	a_{21} x_1 + a_{22} x_2 + a_{23} x_3 & = & y_2 \\
	a_{31} x_1 + a_{32} x_2 + a_{33} x_3 & = & y_3
\end{eqnarray}
Note that each of the coefficients has the same variable name, $a$, but 
different subscripts.
The first subscript increments with the equation while the second increments
based on which $x$ the $a$ multiplies.
A similar system is used for the constants.

Lets rewrite this again, but for an arbitrarily large system:
\begin{eqnarray}
	a_{11} x_1 + a_{12} x_2 + a_{13} x_3 + ... + a_{1n} x_n & = & y_1 \\
	a_{21} x_1 + a_{22} x_2 + a_{23} x_3 + ... + a_{2n} x_n & = & y_2 \\
	a_{31} x_1 + a_{32} x_2 + a_{33} x_3 + ... + a_{3n} x_n & = & y_3 \\ \nonumber
... \\
a_{m1} x_1 + a_{m2} x_2 + a_{m3} x_3 + ... + a_{mn} x_n & = & y_m
\end{eqnarray}
Note that there are $m$ equations and $n$ unknowns.
Now we rewrite it again using a more powerful technique called 
{\it summation notation}:
\begin{equation}
\sum_{j=1}^n a_{ij} x_j = y_i; ~ i=[1..m]
\end{equation}
Notice how much more compact this notation is yet nonetheless captures
everything contained in the previous two systems.
We can encapsulate this even more by {\it abstracting} the entire operation
as a type of multiplication:
\begin{equation}
A \vec x = \vec y
\label{matrix_vector_multiplication}
\end{equation}
where $A$ is an $m \times n$ {\it array}, $\vec y$ is a {\it vector} of length
$m$ and $\vec x$ is a vector of length $n$.
To bring this full circle, we write:
\begin{equation}
A = \left [ \begin{array}{ccc}
1 & 2 & 3 \\
3 & 7 & 2 \\
2.1 & 5.3 & 4.7
\end{array}
\right ]
\end{equation}
and:
\begin{equation}
\vec y = \left [ \begin{array}{l}
4 \\
5 \\
0.5
\end{array}
\right ]
\end{equation}

Half the battle in mastering linear algebra comes in mastering the notation.
Usually when I am stuck in a problem I will move down to each of the less
abstract notations in turn, which, while more verbose, are also more clear.

\subsection{Example: proving the distributive property}

The same notation is used in Equation (\ref{matrix_vector_multiplication})
as for multiplication.
Indeed, it is even called multiplication: we are multiplying a matrix with
a vector.
There is a reason for this.
Matrix multiplication shares many of the same properties as scalar multiplication.
One of these is the distributive property:
\begin{equation}
A \vec x + A \vec y = A (\vec x + \vec y)
\end{equation}
We can use summation notation to prove this property:
\begin{equation}
\sum_j a_{ij} x_j + \sum_j a_{ij} y_j = \sum_j a_{ij} (x_j + y_j)
\end{equation}

\subsection{Matrix transpose}

One way in which matrix multiplication differs from scalar multiplication
is that it is not commutative.
In fact we can write:
\begin{equation}
A^T \vec x = \vec x A
\end{equation}
where $A^T$ is the {\it transpose} of matrix $A$.
Writing this out in summation notation makes it clear.
We simply swap the roles of $i$ and $j$:
\begin{equation}
\sum_j a_{ji} x_j
\end{equation}
Or alternatively:
\begin{equation}
\sum_i a_{ij} x_i
\end{equation}

\section{Solving systems of linear equations}

\label{solving}

In order to solve for the system in (\ref{example_system}) we need to learn
one more notation system.
This one removes all the variables leaving only the constants and coefficients:
\begin{equation}
	\left [ \begin{array}{ccc|c}
			1 & 2 & 3 & 4 \\
			3 & 7 & 2 & 5 \\
			2.1 & 5.3 & 4.7 & 0.5
	\end{array} \right ]
\end{equation}
There are three elementary operations we can do on the rows of this system.
As we would expect,
any operation done on a row of coefficients is repeated on the constant.
\begin{enumerate}
	\item Multiply a row by a constant value
	\item Add two rows
	\item Exchange rows
\end{enumerate}

We start by multiplying the second row by $1/3$:
\begin{equation}
	\left [ \begin{array}{ccc|c}
			1 & 2 & 3 & 4 \\
			1 & 7/3 & 2/3 & 5/3 \\
			2.1 & 5.3 & 4.7 & 0.5
	\end{array} \right ]
\end{equation}
This allows us to subtract the first row from the second:
\begin{equation}
	\left [ \begin{array}{ccc|c}
			1 & 2 & 3 & 4 \\
			0 & 1/3 & -7/3 & -7/3 \\
			2.1 & 5.3 & 4.7 & 0.5
	\end{array} \right ]
\end{equation}
thus eliminating the element in the second row, first column.
We repeat this procedure for the third row:
\begin{equation}
	\left [ \begin{array}{ccc|c}
			1 & 2 & 3 & 4 \\
			0 & 1/3 & -7/3 & -7/3 \\
			1 & 53/21 & 47/21 & 5/21
	\end{array} \right ]
\end{equation}
\begin{equation}
	\left [ \begin{array}{ccc|c}
			1 & 2 & 3 & 4 \\
			0 & 1/3 & -7/3 & -7/3 \\
			0 & 3/7 & -16/21 & -79/21
	\end{array} \right ]
\end{equation}
Now we multiply the second row by $9/7$:
\begin{equation}
	\left [ \begin{array}{ccc|c}
			1 & 2 & 3 & 4 \\
			0 & 3/7 & -63/21 & -63/21 \\
			0 & 3/7 & -16/21 & -79/21
	\end{array} \right ]
\end{equation}
(To make it easier for the next step we haven't reduced all the fractions...)
and subtract it from the last row:
\begin{equation}
	\left [ \begin{array}{ccc|c}
			1 & 2 & 3 & 4 \\
			0 & 3/7 & -3 & -3 \\
			0 & 0 & -79/21 & -16/21
	\end{array} \right ]
\end{equation}
The matrix is now in {\it upper triangular form} which makes it easy to
solve:
\begin{eqnarray}
	-79/21 x_3 & = & -16/21 \\
	x_3 & = & 21/79
\end{eqnarray}
\begin{eqnarray}
	3/7 x_2 - 3 x_1 & = & -3 \\
	3/7 x_2 - 3 (21/79) & = & -3 \\
	x_2 & = & - 406/79
\end{eqnarray}
\begin{eqnarray}
	1 x_1 + 2 x_2 + 3 x_3 & = & 4 \\
	1 x_1 + 2 (- 406/79) + 3 (21/79) & = & 4 \\
	x_1 & = & 1065/79
\end{eqnarray}

The process can also be continued until the matrix is {\it diagonal}:
\begin{equation}
	\left [ \begin{array}{ccc|c}
			1 & 0 & 0 & 1065/79 \\
			0 & 1 & 0 & -406/79 \\
			0 & 0 & 1 & 1065/79
	\end{array} \right ]
\end{equation}
Overall the method is known as {\it Gaussian elimination} or
{\it row reduction}.

\section{Matrix multiplication and the inverse}

Here we define matrix multiplication using the summation
notation introduced in Section \ref{notation}.
Let $A$ be an $m \times p$ matrix while $B$ is a $p \times n$ matrix:
\begin{equation}
	C = A B
\end{equation}
\begin{equation}
	c_{ij} = \sum_{k=1}^p a_{ik} b_{kj}| i=[1..m]; j=[1..n]
\end{equation}
Note that summation is over the {\it inside} indices thus the inside dimensions
of the two matrices must agree.
The product, $C$, must be an $m \times n$ matrix.

The {\it identity matrix} is defined as a diagonal matrix in which all the
non-zero (diagonal) elements are ones.
Here is the $3 \times 3$ identity:
\begin{equation}
	\left [ \begin{array}{ccc}
			1 & 0 & 0 \\
			0 & 1 & 0 \\
			0 & 0 & 1 
	\end{array} \right ]
\end{equation}
The inverse of a square matrix returns the identity when multiplied with
the original matrix:
\begin{equation}
	A^{-1} A = A A^{-1} = I
\end{equation}

To calculate the inverse, perform Gaussian elimination, as above, except
using the identity matrix as the right-hand-side:
\begin{equation}
	\left [ \begin{array}{ccc|ccc}
			1 & 2 & 3 & 1 & 0 & 0 \\
			3 & 7 & 2 & 0 & 1 & 0 \\
			2.1 & 5.3 & 4.7 & 0 & 0 & 1
	\end{array} \right ]
\end{equation}

\section{Determinants}

The determinant of a square matrix, $A$, is defined recursively as follows:
\begin{equation}
	\det (A) = | A | = \sum_i (-1)^{i+j} a_{ij} | A_{ij} |
\end{equation}
where the notation, $A_{ij}$,
denotes the sub-matrix formed by removing the $i$th row and $j$th column
from $A$. 
The determinant of a $1\times 1$ matrix is the value of its single element.

Based on this definition we can write the determinant of a $3 \times 3$ matrix:
\begin{eqnarray}
	& \left | \begin{array}{ccc}
		a_{11} & a_{12} & a_{13} \\
		a_{21} & a_{22} & a_{23} \\
		a_{31} & a_{32} & a_{33}
	\end{array} \right | \\ \nonumber
	& = a_{11} 
	\left | \begin{array}{cc} a_{22} & a_{23} \\
	a_{32} & a_{33} \end{array} \right |
	- a_{12} 
	\left | \begin{array}{cc} a_{21} & a_{23} \\
	a_{31} & a_{33} \end{array} \right |
	+ a_{13}
	\left | \begin{array}{cc} a_{21} & a_{22} \\
	a_{31} & a_{32} \end{array} \right | \\ \nonumber
	       & = a_{11} (a_{22} a_{33} - a_{23} a_{32})
	- a_{12} (a_{21} a_{33} - a_{23} a_{31})
	+ a_{13} (a_{21} a_{32} - a_{22} a_{31})
\end{eqnarray}
and use this formula to calculate the determinant of the example matrix:

Here we prove the most important property of the determinant.
Going back to the three matrix operations listed at the beginning of Section
\ref{solving}: note that each of these operations can be represented 
themselves as matrices or {\it elementary matrices}.
Consider the first two transformations:
\begin{equation}
	\left [ \begin{array}{ccc}
			1 & 0 & 0 \\
			0 & -1/3 & 0 \\
			0 & 0 & 1 
	\end{array} \right ]
\end{equation}
(note that subtraction is equivalent to multiplication with -1 followed by
addition) and:
\begin{equation}
	\left [ \begin{array}{ccc}
			1 & 0 & 0 \\
			1 & 1 & 0 \\
			0 & 0 & 1 
	\end{array} \right ]
\end{equation}
To swap the first and second rows, for instance, we would use:
\begin{equation}
	\left [ \begin{array}{ccc}
			0 & 1 & 0 \\
			1 & 0 & 0 \\
			0 & 0 & 1 
	\end{array} \right ]
\end{equation}
From this it is apparent that any matrix can be built up from the product
of a set of elementary matrices.
The determinant of the first elementary matrix is $-1/3$ while that of the 
second one is $1$ and the third, $-1$.
It is easy to show that the determinant of the product of any two elementary
matrices is the same as the product of the determinants.
Since any matrix can be built up from the product of elementary matrices,
this property must be true in general, that is, the determinant of the
product is the product of the determinants:
\begin{equation}
	|A B C ... | = |A||B||C|...
\end{equation}
As a corollary to this theorem, a matrix only has an inverse if the 
determinant is non-zero. 
A matrix with non inverse is called a {\it singular matrix} and if substituted
into equation (\ref{matrix_vector_multiplication}), 
will not have a unique solution.
We will need this principle in the next section.

\section{Eigenvectors and eigenvalues}

The eigenvalue equation is as follows:
\begin{equation}
	A \vec v = \lambda \vec v
	\label{eigenequation}
\end{equation}
where $\vec v$ is an {\it eigenvector} and {\it $\lambda$} is an eigenvalue.
In other words, an eigenvector of a matrix is a vector that, when multiplied
with the matrix, returns itself times a constant.
We start by rearranging Equation (\ref{eigenequation}) as follows:
\begin{equation}
	\left ( A - \lambda I \right ) \vec v = 0
	\label{eigenequation2}
\end{equation}
In order for a {\it non-trivial} solution (not all values are zero) to exist,
the determinant of the expression in brackets must be zero:
\begin{equation}
	\left | A - \lambda I \right | = 0
\end{equation}

Lets calculate the eigenvalues of the example matrix:
\begin{eqnarray}
	& \left | \begin{array}{ccc}
1 - \lambda & 2 & 3 \\
3 & 7 - \lambda & 2 \\
2.1 & 5.3 & 4.7 - \lambda
\end{array}
\right | \\ \nonumber
& = (1 - \lambda) \left [(7 - \lambda)(4.7 - \lambda) - (2)(5.3) \right ] -
(2) \left [(3)(4.7 - \lambda) - (2)(2.1) \right ] + (3) \left [(3)(5.3) - (7 - \lambda) (2.1) \right ] \\ \nonumber
\end{eqnarray}
Multiplying through and gathering like terms produces the following cubic
equation:
\begin{equation}
	\lambda^3 - 11.7 \lambda^2 + 0.6 \lambda + 6.1 = 0
\end{equation}
This is known as the {\it characteristic polynomial} of the matrix
and its roots are the eigenvalues:
\begin{equation}
\lambda \approx \left \lbrace \begin{array}{c} 11.6 \\ 0.775 \\ -0.678 \end{array}\right .
\end{equation}

By substituting the resulting eigenvalues back into the bracketed expression
in (\ref{eigenequation2}), the eigenvectors may be derived.
We demonstrate this for the largest eigenvalue:
\begin{equation}
	\left [ \begin{array}{ccc}
		1 - (11.6) & 2 & 3 \\
		3 & 7 - (11.6) & 2 \\
		2.1 & 5.3 & 4.7 - (11.6)
\end{array}
\right ]
=
	\left [ \begin{array}{ccc}
		 -10.6 & 2 & 3 \\
		3 & -4.6 & 2 \\
		2.1 & 5.3 & -7.6)
\end{array}
\right ]
\end{equation}

\end{document}

